\begin{figure} \centering
\begin{subfigure}[b]{0.4\linewidth}
\centering
\resizebox{.75\textwidth}{!}{
\begin{tikzpicture}[mynode/.style={draw,circle, fill=red!10,minimum width=10mm,align=center},
                    mynodex/.style={draw,circle, minimum width=10mm,align=center}]
\node[mynode] (y_1) {$y_1$};
\node[mynode,right =of y_1] (y_2) {$y_2$};
\node[right =of y_2] (y_dots) {$\ldots$};
\node[mynode,right =of y_dots] (y_n) {$y_n$};

\node[mynode,above = of y_1] (x_1) {$x_1$};
\node[mynode,right =of x_1] (x_2) {$x_2$};
\node[right =of x_2] (x_dots) {$\ldots$};
\node[mynode,right =of x_dots] (x_n) {$x_n$};

\node[mynodex, below=of y_2] (beta) {$\vec{\beta}$};
\node[mynodex, right=of beta] (sigma) {$\sigma$};

\foreach \value in {1,2,n}{
  \draw[->] (x_\value) to (y_\value);
  \draw[->] (beta) to (y_\value);
  \draw[->] (sigma) to (y_\value);
}
\end{tikzpicture}
}
\subcaption{}
\end{subfigure} %
\hspace{10mm}
\begin{subfigure}[b]{0.4\linewidth}
\centering
\resizebox{.75\textwidth}{!}{
\begin{tikzpicture}[mynode/.style={draw,circle, fill=red!10,minimum width=10mm,align=center},mynodex/.style={draw,circle, minimum width=10mm,align=center}]

\node[mynode] (y_1) {$y_d$};
\node[mynode,above = of y_1] (x_1) {$x_d$};
\node[right=5mm of y_1] (i) {$d \in 1\ldots n$};

\node[mynodex, below=15mm of y_1] (beta) {$\vec{\beta}$};
\node[mynodex, right=15mm of beta] (sigma) {$\sigma$};

\node[draw,rounded corners=2mm,inner sep=5mm,fit=(y_1) (x_1) (i)] {};

\foreach \value in {x_1,beta,sigma}
\draw[->] (\value) to (y_1);
%\draw[->] (sigma) to (y_1);

\end{tikzpicture}
}
\subcaption{}
\label{fig:b}    
\end{subfigure} 
\caption{Two equivalent Bayesian network diagrams representing a normal linear model with one predictor variable. Diagram b) uses a compact plate notation whereby all variables within the plate are repeated for all values of the index $i$, which takes values from $1$ to $n$.}
\label{fig:bda_sleepstudy_j}
\end{figure}
